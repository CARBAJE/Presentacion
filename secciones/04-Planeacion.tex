\section{Planeación}

\begin{frame}{Juicio de Expertos}
  \begin{columns}[T]
    \column{0.5\textwidth}
    \textbf{M.C.José Alberto Torres León}
    \begin{itemize}
      \item Modificación de parámetros en tiempo de ejecución
      \item Simulaciones apegadas a la realidad
      \item Interacciones dinámicas
    \end{itemize}   
    \column{0.5\textwidth}
    \textbf{Dr. Daniel Molina Pérez}
       \begin{itemize}
         \item WH-fast idóneo para determinar trayectorias.
         \item El uso de algoritmos bioinspirados está justificado.
         \item Aplicaciones aeroespaciales y astrofísicas.
       \end{itemize}
   \end{columns}
 \end{frame}
  
%   \RaggedRight%
%   \scriptsize
%   Egresado del Centro de Investigación en Cómputo (CIC) del Instituto Politécnico Nacional, sus áreas de especialización
%   incluyen agentes inteligentes para videojuegos (jugadores y dinámicos), optimización evolutiva entre otros.

%   \bigskip

%   \textbf{Resumen de Opiniones y Observaciones Obtenidas}
%   \RaggedRight%
%   \scriptsize
%   Tras la revisión del material y la discusión, el M.C. Torres León aportó las siguientes observaciones y valoraciones:
  
% \end{frame}

% \begin{frame}{Juicio de Expertos: Dr. Daniel Molina Pérez}
%   \textbf{Perfil del Experto}
%   \RaggedRight
%   \scriptsize
%   Investigador titular en Mecatrónica del CIDETEC-IPN, con Doctorado en Ingeniería de Sistemas Robóticos y Mecatrónica. Su experiencia se centra en algoritmia, algoritmos bioinspirados, optimización (aplicada y evolutiva), lo que lo convierte en un validador experto para este proyecto.

%   \bigskip

%   \textbf{Resumen de Opiniones y Observaciones Obtenidas}
%   \RaggedRight
%   \scriptsize
%   Tras la revisión del material y la discusión, el M.C. Torres León aportó las siguientes observaciones y valoraciones:
%   \begin{itemize}
%     \item WH-fast idóneo para determinar trayectorias.
%     \item El uso de algoritmos bioinspirados está justificado.
%     \item Aplicaciones aeroespaciales y astrofísicas.
%   \end{itemize}
% \end{frame}

% \begin{frame}{Requerimientos funcionales}
%   \begin{columns}[T]
%     \column{0.5\textwidth}
%       \begin{itemize}
%         \item Iniciar simulación dinámica de masa.
%         \item Simular interacciones gravitacionales.
%         \item Usar método multipolar rápido y Barnes-Hut.
%       \end{itemize}
%     \column{0.5\textwidth}
%       \begin{itemize}
%         \item Implementar algoritmos bioinspirados para ajustar dinámicamente los parametros.
%         \item Incluir módulo de visualización gráfica de evolución.
%         \item Ofrecer interfaz básica para modificar parámetros.
%       \end{itemize}
%   \end{columns}
% \end{frame}

\begin{frame}{Requerimientos Funcionales}
  \vspace{-0.5cm}
  \begin{table}[H]
      \centering
      \caption{Requerimientos Funcionales}
      \vspace{-0.35cm}
      \begin{adjustbox}{max width=1\textwidth,max height=0.4\textheight, keepaspectratio}
      \begin{tabular}{p{1cm}p{5cm}p{1.75cm}p{2.25cm}p{5cm}}
      \toprule %chktex 44
      \textbf{ID} & \textbf{Descripción} & \textbf{Prioridad} & \textbf{Actor} & \textbf{Criterios de Aceptación} \\
      \midrule
      RF-001 & Modificación dinámica de la masa durante la simulación. & Alta & \seqsplit{Usuario/Modelo} & El usuario cambia la masa y la simulación responde adecuadamente. \\
      \midrule
      RF-002 & Simulación de interacciones gravitacionales entre dos cuerpos. & Alta & Modelo & Se calculan y representan correctamente las fuerzas gravitacionales. \\
      \midrule
      RF-003 & Uso de FMM y Barnes-Hut para optimización. & Media & Modelo & Se implementan y mejoran el rendimiento frente a métodos directos. \\
      \midrule
      RF-004 & Ajuste dinámico de parámetros mediante algoritmos bioinspirados. & Alta & Modelo & Mantienen la estabilidad del sistema durante cambios. \\
      \midrule
      RF-005 & Visualización gráfica de la evolución de los cuerpos. & Media & Usuario & Muestra trayectorias y estados claramente durante la simulación. \\
      \midrule
      RF-006 & Interfaz básica para modificar parámetros y ver resultados. & Media & Usuario & El usuario ajusta parámetros y observa los efectos. \\
      \bottomrule
      \end{tabular}
      \end{adjustbox}
  \end{table}
\end{frame}

% \begin{frame}{Requerimientos funcionales}
%   \begin{columns}[T]
%     \column{0.5\textwidth}
%       \begin{itemize}
%         \item Iniciar simulación dinámica de masa.
%         \item Simular interacciones gravitacionales.
%         \item Usar método multipolar rápido y Barnes-Hut.
%       \end{itemize}
%     \column{0.5\textwidth}
%       \begin{itemize}
%         \item Implementar algoritmos bioinspirados para ajustar dinámicamente los parametros.
%         \item Incluir módulo de visualización gráfica de evolución.
%         \item Ofrecer interfaz básica para modificar parámetros.
%       \end{itemize}
%   \end{columns}
% \end{frame}

\begin{frame}{Requerimientos No Funcionales}
  \vspace{-0.4cm}
  \begin{table}[H]
      \centering
      \caption{Requerimientos No Funcionales}
      \vspace{-0.35cm}
      \begin{adjustbox}{max width=1\textwidth, max height=0.4\textheight, keepaspectratio}
      \begin{tabular}{p{1.2cm}p{5cm}p{1.75cm}p{2.25cm}p{5cm}}
      \toprule
      \textbf{ID} & \textbf{Descripción} & \textbf{Prioridad} & \textbf{Actor} & \textbf{Criterios de Aceptación} \\
      \midrule
      RNF-001 & Optimización de la simulación para eficiencia computacional. & Alta & Modelo & Ejecuta en tiempo razonable y sin alto consumo de recursos. \\
      \midrule
      RNF-002 & Soporte a futuro para más de dos cuerpos. & Media & Modelo & Escalable sin cambios estructurales mayores. \\
      \midrule
      RNF-003 & Respeto a leyes físicas en interacciones gravitacionales. & Alta & Modelo & Resultados coherentes con la física. \\
      \midrule
      RNF-004 & Estabilidad ante modificaciones dinámicas de parámetros. & Alta & Modelo & No colapsa ni muestra comportamientos erráticos. \\
      \midrule
      RNF-005 & Interfaz intuitiva y accesible para usuarios no técnicos. & Media & Usuario & Puede usarse sin conocimientos avanzados. \\
      \midrule
      RNF-006 & Ejecución en hardware de gama media. & Media & Modelo & Funciona en equipos con especificaciones mínimas. \\
      \midrule
      RNF-007$^{*}$ & Integración con entornos como Unreal Engine. & Baja & Modelo & Integración funcional con motores externos. \\
      \midrule
      RNF-008 & Arquitectura modular para facilitar mantenimiento. & Media & Desarrollador & Componentes separados y modificables. \\
      \midrule
      RNF-009 & Documentación clara para usuarios y desarrolladores. & Media & Desarrollador & Completa y comprensible para todos los perfiles. \\
      \bottomrule
      \end{tabular}
      \vspace{0.5em}
      \end{adjustbox}
      \begin{minipage}{\textwidth}
    \footnotesize{\tiny $^{*}$ El requerimiento no funcional No. 07 (RNF-007), solo hace mención a la idea conceptual del requerimiento, no se plantea su elaboración.}
    \end{minipage}
  \end{table}

\end{frame}

% \begin{frame}{Requerimientos no funcionales}
%   \begin{columns}[T]
%     \column{0.3\textwidth}
%       \begin{itemize}
%         \item Optimizar complejidad de simulaciones.
%         \item Permitir extensión futura a más de dos cuerpos.
%         \item Respetar leyes de la física y mecánica celeste.
%       \end{itemize}
%     \column{0.3\textwidth}
%       \begin{itemize}
%         \item Mantener simulaciones estables al modificar parámetros.
%         \item Ser interfaz intuitiva y accesible.
%         \item Ejecutarse en hardware de gama media.
%       \end{itemize}
%     \column{0.3\textwidth}
%       \begin{itemize}
%         \item Integrarse con entornos virtuales (Unreal Engine).
%         \item Ser modular para facilitar actualizaciones y correcciones.
%         \item Incluir documentación clara.
%       \end{itemize}
%   \end{columns}
% \end{frame}

\begin{frame}{Formulación del Problema de Optimización}

\begin{block}{Función Objetivo}
  Minimizar el valor absoluto del exponente de Lyapunov ($\lambda$) en función de las masas $m_1$ y $m_2$:
  $$ \text{Minimizar} \quad |\lambda(m_1, m_2)| $$ %chktex 45
\end{block}

\begin{block}{Variables de Decisión}
  \begin{itemize}
      \item $m_1$: Masa del cuerpo 1
      \item $m_2$: Masa del cuerpo 2
  \end{itemize}
\end{block}
\end{frame}

\begin{frame}{Formulación del Problema de Optimización}
\begin{block}{Restricciones}
  \begin{itemize}
      \item El exponente de Lyapunov debe ser menor a $\alpha$:
      $$ \lambda(m_1, m_2) < \alpha $$ %chktex 45
      \item La masa del cuerpo 2 debe ser menor a $\mu$ veces la masa del cuerpo 1:
      $$ m_2 < \mu m_1 $$ %chktex 45
      \item Las masas deben ser positivas*:
      $$ m_1 > 0 $$ %chktex 45
      $$ m_2 > 0 $$ %chktex 45
  \end{itemize}
  \vspace{0cm}
  \tiny{* Llamamos masas a las variables que busca el AG;\ como explora todo $\mathbb{R}$, se debe restringir su dominio para limitar la búsqueda a valores factibles para el sistema.}
\end{block}

\end{frame}