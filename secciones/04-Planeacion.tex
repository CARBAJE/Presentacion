\section{Planeación}

\begin{frame}{Juicio de Expertos: M.C. José Alberto Torres León}
  \textbf{Perfil del Experto}
  \RaggedRight
  \scriptsize
  Egresado del Centro de Investigación en Cómputo (CIC) del Instituto Politécnico Nacional, sus áreas de especialización
  incluyen agentes inteligentes para videojuegos (jugadores y dinámicos), optimización evolutiva entre otros.

  \bigskip

  \textbf{Resumen de Opiniones y Observaciones Obtenidas}
  \RaggedRight
  \scriptsize
  Tras la revisión del material y la discusión, el M.C. Torres León aportó las siguientes observaciones y valoraciones:
  \begin{itemize}
    \item Modificación de parámetros en tiempo de ejecución
    \item Simulaciones apegadas a la realidad
    \item Interacciones dinámicas
  \end{itemize}
\end{frame}

\begin{frame}{Juicio de Expertos: Dr. Daniel Molina Pérez}
  \textbf{Perfil del Experto}
  \RaggedRight
  \scriptsize
  Investigador titular en Mecatrónica del CIDETEC-IPN, con Doctorado en Ingeniería de Sistemas Robóticos y Mecatrónica.
  Su experiencia se centra en algoritmia, algoritmos bioinspirados, optimización (aplicada y evolutiva).

  \bigskip

  \textbf{Resumen de Opiniones y Observaciones Obtenidas}
  \RaggedRight
  \scriptsize
  Tras la revisión del material y la discusión, el Dr. Molina Pérez aportó las siguientes observaciones y valoraciones:
  \begin{itemize}
    \item Aplicaciones aeroespaciales y astrofísicas
    \item El uso de alboritmos bioinspirados está justificado
    \item WH-Fast idóneo para determinar trayectorias
  \end{itemize}
\end{frame}

\begin{frame}{Requerimientos funcionales}
  \begin{columns}[T]
    \column{0.5\textwidth}
      \begin{itemize}
        \item Iniciar simulación dinámica de masa.
        \item Simular interacciones gravitacionales.
        \item Usar método multipolar rápido y Barnes-Hut.
      \end{itemize}
    \column{0.5\textwidth}
      \begin{itemize}
        \item Implementar algoritmos bioinspirados para ajustar dinámicamente los parametros.
        \item Incluir módulo de visualización gráfica de evolución.
        \item Ofrecer interfaz básica para modificar parámetros.
      \end{itemize}
  \end{columns}
\end{frame}

\begin{frame}{Requerimientos no funcionales}
  \begin{columns}[T]
    \column{0.3\textwidth}
      \begin{itemize}
        \item Optimizar complejidad de simulaciones.
        \item Permitir extensión futura a más de dos cuerpos.
        \item Respetar leyes de la física y mecánica celeste.
      \end{itemize}
    \column{0.3\textwidth}
      \begin{itemize}
        \item Mantener simulaciones estables al modificar parámetros.
        \item Ser interfaz intuitiva y accesible.
        \item Ejecutarse en hardware de gama media.
      \end{itemize}
    \column{0.3\textwidth}
      \begin{itemize}
        \item Integrarse con entornos virtuales (Unreal Engine).
        \item Ser modular para facilitar actualizaciones y correcciones.
        \item Incluir documentación clara.
      \end{itemize}
  \end{columns}
\end{frame}

\begin{frame}{Formulación del Problema de Optimización}

\begin{block}{Función Objetivo}
  Minimizar el valor absoluto del exponente de Lyapunov ($\lambda$) en función de las masas $m_1$ y $m_2$:
  $$ \text{Minimizar} \quad |\lambda(m_1, m_2)| $$
\end{block}

\begin{block}{Variables de Decisión}
  \begin{itemize}
      \item $m_1$: Masa del cuerpo 1
      \item $m_2$: Masa del cuerpo 2
  \end{itemize}
\end{block}
\end{frame}

\begin{frame}{Formulación del Problema de Optimización}
\begin{block}{Restricciones}
  \begin{itemize}
      \item El exponente de Lyapunov debe ser menor a 0.1:
      $$ \lambda(m_1, m_2) < 0.1 $$
      \item La masa del cuerpo 2 debe ser menor a 10 veces la masa del cuerpo 1:
      $$ m_2 < 10 m_1 $$
      \item Las masas deben ser positivas:
      $$ m_1 > 0 $$
      $$ m_2 > 0 $$
  \end{itemize}
\end{block}

\end{frame}