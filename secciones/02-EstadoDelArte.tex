\section{Estado del Arte}

\begin{frame}{Estado del arte}
    \centering
    \captionof{table}{Comparativa contra soluciones disponibles}%
    \label{tab:arte}
    \vspace{-0.1cm}
    \begin{adjustbox}{max width=0.9\textwidth,max height=0.7\textheight, keepaspectratio}
        \renewcommand{\arraystretch}{1.3}
            \begin{tabular}{@{}>{\bfseries}p{0.35\textwidth} p{0.4\textwidth} p{0.20\textwidth} p{0.15\textwidth} p{0.35\textwidth}@{}}
            \toprule
            \textbf{Producto o metodo} & \textbf{Características} & \textbf{Escalabilidad} & \textbf{Usa IA} & \textbf{Cambios dinamicos} \\
            \midrule
            ode\_num\_int & Framework C++11 modular para EDOs; orientado a pruebas de integradores. & Media & No & No \\
            Representación Geométrica & Quadtrees/Octrees para geometría eficiente, no simula dinámica. & Alta & No & No \\
            Método n-NNN & Simulación con n-vecinos y cirugía Hamiltoniana; inspirado en IA.\ & Alta & Sí & No \\
            PKDGRAV3 & Hidrodinámica sin malla (MFM/MFV); vecinos adaptativos. & Alta & No & No \\
            \bottomrule
            \end{tabular}
    \end{adjustbox}
    \smallskip
\end{frame}



\begin{frame}{Estado del arte}
    \centering
    \captionof{table}{Comparativa contra soluciones disponibles}%
    \label{tab:arte}
    \vspace{-0.1cm}
    \begin{adjustbox}{max width=0.9\textwidth,max height=0.7\textheight, keepaspectratio}
        \renewcommand{\arraystretch}{1.3}
            \begin{tabular}{@{}>{\bfseries}p{0.35\textwidth} p{0.4\textwidth} p{0.20\textwidth} p{0.15\textwidth} p{0.35\textwidth}@{}}
            \toprule
            \textbf{Producto o metodo} & \textbf{Características} & \textbf{Escalabilidad} & \textbf{Usa IA} & \textbf{Cambios dinamicos} \\
            \midrule
            SPH/N-body Híbrido & Interacciones gas-estrella con árbol Barnes-Hut y pasos bloque. & Alta & No & No \\
            Integrador Simpléctico & Orden 2+, reversible; ideal para colisiones. & Media & No & No \\
            Solver TPM & Combina PM y Tree según densidad local; altamente paralelizado. & Alta & No & No \\
            Algoritmo TPM & Descomposición por densidad; integración multi-escala. & Alta & No & No \\
            \bottomrule
            \end{tabular}
    \end{adjustbox}
    \smallskip
\end{frame}


\begin{frame}{Estado del arte}
    \centering
    \captionof{table}{Comparativa contra soluciones disponibles}%
    \label{tab:arte}
    \vspace{-0.1cm}
    \begin{adjustbox}{max width=0.9\textwidth,max height=0.7\textheight, keepaspectratio}
        \renewcommand{\arraystretch}{1.3}
            \begin{tabular}{@{}>{\bfseries}p{0.35\textwidth} p{0.4\textwidth} p{0.20\textwidth} p{0.15\textwidth} p{0.35\textwidth}@{}}
            \toprule
            \textbf{Producto o metodo} & \textbf{Características} & \textbf{Escalabilidad} & \textbf{Usa IA} & \textbf{Cambios dinamicos} \\
            \midrule
            REBOUND & Modular; incluye varios integradores, colisiones y condiciones frontera. & Alta & No & No \\
            Estabilidad Planetaria & Estudio con REBOUND;\ Gini vs.\ inestabilidad. No eventos internos. & Alta & No & No \\
            \midrule
            \rowcolor{yellow!20}
            \textbf{Solución Propuesta} & \textbf{Combina FMM/Barnes-Hut para cálculo gravitacional eficiente con Algoritmos Bioinspirados para la \textit{optimización y ajuste dinámico} de parámetros (masa) durante la simulación.}  & \textbf{Alta} & \textbf{SÍ} & \textbf{SÍ} \\
            \bottomrule
            \end{tabular}
    \end{adjustbox}
    \smallskip
\end{frame}

\begin{frame}{Comparativa de Métodos N-cuerpos}
    \centering
    \captionof{table}{Comparativa de Métodos N-cuerpos}%
    \label{tab:comparativa}
    \vspace{-0.1cm}
    \begin{adjustbox}{max width=1\textwidth,max height=0.4\textheight, keepaspectratio}
        \renewcommand{\arraystretch}{1.3}
            \begin{tabular}{>{
            \raggedright\arraybackslash}p{5.2cm}
            >{\centering\arraybackslash}p{3.2cm}
            >{\centering\arraybackslash}p{3.2cm}
            >{\centering\arraybackslash}p{3.2cm}
            >{\centering\arraybackslash}p{3.5cm}
            }
                \toprule
                \textbf{Método} & \textbf{Alta Escala} & \textbf{Modular} & \textbf{Usa IA} & \textbf{Dinámico*} \\
                \midrule
                Framework ode\_num\_int & \color{red}{\xmark} & \color{green}{\checkmark} & \color{red}{\xmark} & \color{red}{\xmark} \\
                \midrule
                Repr. Geométrica & \color{green}{\checkmark} & \color{red}{\xmark} & \color{red}{\xmark} & \color{red}{\xmark} \\
                \midrule
                Método n-NNN & \color{green}{\checkmark} & \color{red}{\xmark} & \color{green}{\checkmark} & \color{red}{\xmark} \\
                \midrule
                PKDGRAV3 & \color{green}{\checkmark} & \color{red}{\xmark} & \color{red}{\xmark} & \color{red}{\xmark} \\
                \midrule
                Híbrido SPH/N-body & \color{green}{\checkmark} & \color{green}{\checkmark} & \color{red}{\xmark} & \color{red}{\xmark} \\
                \midrule
                Int. Simpléctico & \color{red}{\xmark} & \color{green}{\checkmark} & \color{red}{\xmark} & \color{red}{\xmark} \\
                \midrule
                Solver Híbrido TPM & \color{green}{\checkmark} & \color{red}{\xmark} & \color{red}{\xmark} & \color{red}{\xmark} \\
                \midrule
                Algoritmo TPM & \color{green}{\checkmark} & \color{red}{\xmark} & \color{red}{\xmark} & \color{red}{\xmark} \\
                \midrule
                REBOUND & \color{green}{\checkmark} & \color{green}{\checkmark} & \color{red}{\xmark} & \color{red}{\xmark} \\
                \midrule
                Est. Planetaria & \color{green}{\checkmark} & \color{red}{\xmark} & \color{red}{\xmark} & \color{red}{\xmark} \\
                \midrule
                \rowcolor{yellow!30}
                \textbf{Solución Propuesta} & \textbf{\color{green}{\checkmark}} & \textbf{\color{green}{\checkmark}} & \textbf{\color{green}{\checkmark}} & \textbf{\color{green}{\checkmark}} \\
                \bottomrule
        \end{tabular}
    \end{adjustbox}
    \smallskip
    \vspace{0.3cm}\\
    \tiny{*Dinámico: Capacidad de modificar parámetros clave durante la ejecución}
\end{frame}