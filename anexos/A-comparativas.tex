\section[Comparativas]{Tablas Comparativas del Marco Teórico}

\begin{frame}{Anexo \thesection~: Lenguaje de Programación}
    \centering
    \captionof{table}{Comparativa de Lenguaje de Programación}%
    \label{tab:comparativa_languages}
    \vspace{-0.1cm}
    \begin{adjustbox}{max width=0.9\textwidth, max height=0.4\textheight, keepaspectratio}
        \renewcommand{\arraystretch}{1.1}
        \begin{tabular}{@{}p{7cm} >{\centering\arraybackslash}p{3.5cm} >{\centering\arraybackslash}p{3.5cm}@{}}
            \toprule
            \textbf{Característica} & \textbf{C++} & \textbf{Python} \\
            \midrule
            Velocidad de Ejecución & \color{green}{\checkmark} & \color{red}{\xmark} \\
            \midrule
            Velocidad de Desarrollo & \color{red}{\xmark} & \color{green}{\checkmark} \\
            \midrule
            Disponibilidad de Bibliotecas Científicas & \color{green}{\checkmark} & \color{green}{\checkmark} \\
            \midrule
            Facilidad de Uso de Bibliotecas & \color{red}{\xmark} & \color{green}{\checkmark} \\
            \midrule
            Visualización & \color{red}{\xmark} & \color{green}{\checkmark} \\
            \midrule
            Integración con REBOUND & \color{green}{\checkmark} & \color{green}{\checkmark} \\
            \midrule
            Algoritmos Bioinspirados & \color{green}{\checkmark} & \color{green}{\checkmark} \\
            \midrule
            Control de Gestión de Memoria & \color{green}{\checkmark} & \color{red}{\xmark} \\
            \midrule
            Simplicidad de Gestión de Memoria & \color{red}{\xmark} & \color{green}{\checkmark} \\
            \midrule
            Curva de Aprendizaje & \color{red}{\xmark} & \color{green}{\checkmark} \\
            \midrule
            Prototipado Rápido & \color{red}{\xmark} & \color{green}{\checkmark} \\
            \midrule
            Facilidad de Depuración & \color{red}{\xmark} & \color{green}{\checkmark} \\
            \midrule
            Enfoque del Proyecto & \color{red}{\xmark} & \color{green}{\checkmark} \\
            \bottomrule
        \end{tabular}
    \end{adjustbox}
    \smallskip
\end{frame}


\begin{frame}{Anexo \thesection~: Comparativa de Bibliotecas N-Cuerpos}
    \centering
    \captionof{table}{Comparativa de bibliotecas para simulación N-cuerpos.}%
    \label{tab:nbody-comparativa-beamer}
    \vspace{-0.1cm}
    \begin{adjustbox}{max width=\textwidth, max height=0.35\textheight, keepaspectratio}
        \renewcommand{\arraystretch}{1.1}
        \begin{tabular}{@{}p{5cm} >{\centering\arraybackslash}p{3.5cm} >{\centering\arraybackslash}p{2.5cm} >{\centering\arraybackslash}p{2.5cm} >{\centering\arraybackslash}p{2.5cm} >{\centering\arraybackslash}p{2.5cm}@{}}
            \toprule
            \textbf{Característica} & \textbf{REBOUND} & \textbf{PKDGRAV3} & \textbf{AMUSE} & \textbf{NBody (Python)} & \textbf{PyGaia} \\
            \midrule
            Lenguaje Principal & C (Python API) & C++ & Python & Python & Python \\
            \midrule
            Manejo de Colisiones & \color{green}{\checkmark} & \color{red}{\xmark} & \color{green}{\checkmark} & \color{red}{\xmark} & N/A \\
            \midrule
            Enfoque en Cosmología & \color{red}{\xmark} & \color{green}{\checkmark} & \color{red}{\xmark} & \color{red}{\xmark} & \color{red}{\xmark} \\
            \midrule
            Hidrodinámica (SPH/Gas) & \color{red}{\xmark} & \color{green}{\checkmark} & \color{green}{\checkmark} & \color{red}{\xmark} & \color{red}{\xmark} \\
            \midrule
            Paralelización Avanzada* & \color{green}{\checkmark} & \color{green}{\checkmark} & \color{green}{\checkmark} & \color{red!40!orange}{Parcial} & \color{red}{\xmark} \\
            \midrule
            Flexibilidad/Modularidad** & \color{green}{\checkmark} & \color{red}{\xmark} & \color{green}{\checkmark} & \color{red!40!orange}{Media} & \color{red}{\xmark} \\
            \midrule
            Facilidad de Uso (Python) & \color{green}{\checkmark} & \color{red}{\xmark} & \color{red}{\xmark} & \color{green}{\checkmark} & \color{green}{\checkmark} \\
            \midrule
            Comunidad Activa & \color{green}{\checkmark} & \color{green}{\checkmark} & \color{green}{\checkmark} & \color{red!40!orange}{Variable} & \color{green}{\checkmark} \\
            \midrule
            Soporta 2 Cuerpos & \color{green}{\checkmark} & \color{red}{\xmark} & \color{red!40!orange}{Posible} & \color{green}{\checkmark} & \color{red}{\xmark} \\
            \midrule
            \rowcolor{yellow!30}
            \textbf{Adecuación al Proyecto} & \textbf{\color{green}{\checkmark}} & \textbf{\color{red}{\xmark}} & \textbf{\color{red}{\xmark}} & \textbf{\color{red}{\xmark}} & \textbf{\color{red}{\xmark}} \\
            \bottomrule
        \end{tabular}
    \end{adjustbox}
    \smallskip
    \vspace{0cm}\\
    \tiny{*Paralelización Avanzada: Soporte para MPI/OpenMP.}\\
    \tiny{**Flexibilidad/Modularidad: Capacidad de personalizar o extender fácilmente.}\\
    \tiny{***Soporta 2 Cuerpos: Adecuado para simulaciones simples de pocos cuerpos.}
\end{frame}

\begin{frame}{Anexo \thesection~: Algoritmos de Optimización}
    \centering
    \captionof{table}{Comparativa de Bibliotecas de Optimización}%
    \label{tab:comparativa_GAs}
    \vspace{-0.1cm}
    \begin{adjustbox}{max width=0.9\textwidth, max height=0.4\textheight, keepaspectratio}
        \renewcommand{\arraystretch}{1.1}
        \begin{tabular}{@{}p{3cm} >{\centering\arraybackslash}p{3.5cm} >{\centering\arraybackslash}p{3.5cm} >{\centering\arraybackslash}p{3.5cm} >{\centering\arraybackslash}p{2.5cm}@{}}
            \toprule
            \textbf{Biblioteca} & \textbf{Enfoque Clave} & \textbf{Ideal Para} & \textbf{Complejidad} & \textbf{Adecuación al Proyecto} \\
            \midrule
            \rowcolor{yellow!25}
            \texttt{pymoo} & Multiobjetivo & Problemas complejos & Moderada & \color{green}{\checkmark} \\
            \midrule
            \texttt{DEAP} & Flexibilidad máxima & Experimentación profunda & Alta & \color{red}{\xmark} \\
            \midrule
            \texttt{Platypus} & Multiobjetivo fácil & Implementación rápida & Baja-Moderada & \color{red!40!orange}{\textasciitilde} \\
            \midrule
            \texttt{Inspyred} & Versátil & Exploración de algoritmos & Moderada & \color{red}{\xmark} \\
            \midrule
            \texttt{Nevergrad} & Sin derivadas & Funciones costosas/ruidosas & Moderada & \color{red!40!orange}{\textasciitilde} \\
            \midrule
            \texttt{PyGMO/PaGMO} & Alto rendimiento & Simulaciones costosas & Moderada-Alta & \color{red!40!orange}{\textasciitilde} \\
            \bottomrule
        \end{tabular}
    \end{adjustbox}
\end{frame}

\begin{frame}{Anexo \thesection~: Bibliotecas de Visualización}
    \centering
    \captionof{table}{Comparativa de Bibliotecas de Visualización}%
    \label{tab:comparativa_graphics}
    \vspace{-0.1cm}
    \begin{adjustbox}{max width=\textwidth, max height=0.35\textheight, keepaspectratio}
        \renewcommand{\arraystretch}{1.1}
        \begin{tabular}{@{}p{2.5cm} >{\centering\arraybackslash}p{3cm} >{\centering\arraybackslash}p{2.5cm} >{\centering\arraybackslash}p{2.5cm} >{\centering\arraybackslash}p{2.5cm} >{\centering\arraybackslash}p{2.5cm} >
        {\centering\arraybackslash}p{2.5cm} >{\centering\arraybackslash}p{3.5cm}@{}}
            \toprule
            \textbf{Biblioteca} & \textbf{Uso Principal} & \textbf{Animación} & \textbf{Gráficos Estáticos} & \textbf{Interactividad} & \textbf{Trayectorias 2D/3D} & \textbf{Comunidad/Docs} & \textbf{Adecuación al Proyecto} \\
            \midrule
            \rowcolor{yellow!25}
            \textbf{Matplotlib} & Científicos & \color{green}{\checkmark} & \color{green}{\checkmark} & \color{red!40!orange}{\textasciitilde} & \color{green}{\checkmark} & \color{green}{\checkmark} & \color{green}{\checkmark} \\
            \midrule
            \textbf{Plotly} & Web Interactivo & \color{red!40!orange}{\textasciitilde} & \color{red!40!orange}{\textasciitilde} & \color{green}{\checkmark} & \color{green}{\checkmark} & \color{red}{\xmark} & \color{red!40!orange}{\textasciitilde} \\
            \midrule
            \textbf{Bokeh} & Web Interactivo & \color{red!40!orange}{\textasciitilde} & \color{red!40!orange}{\textasciitilde} & \color{green}{\checkmark} & \color{red!40!orange}{\textasciitilde} & \color{red!40!orange}{\textasciitilde} & \color{red!40!orange}{\textasciitilde} \\
            \midrule
            \textbf{Seaborn} & Estadísticas & \color{red}{\xmark} & \color{red}{\xmark} & \color{red!40!orange}{\textasciitilde} & \color{red}{\xmark} & \color{red!40!orange}{\textasciitilde} & \color{red}{\xmark} \\
            \bottomrule
        \end{tabular}
    \end{adjustbox}
\end{frame}

\begin{frame}{Anexo \thesection~: Comparativa de Bibliotecas GUI}
    \centering
    \captionof{table}{Comparativa de bibliotecas Python para Interfaces Gráficas.}%
    \label{tab:gui-comparativa-beamer}
    \vspace{-0.1cm}
    \begin{adjustbox}{max width=\textwidth, max height=0.8\textheight, keepaspectratio}
        \renewcommand{\arraystretch}{1.1}
        \begin{tabular}{@{}p{3.5cm} >{\centering\arraybackslash}p{2.5cm} >{\centering\arraybackslash}p{2.5cm} >{\centering\arraybackslash}p{2.5cm} >{\centering\arraybackslash}p{2.5cm} >{\centering\arraybackslash}p{2.5cm} >{\centering\arraybackslash}p{2.5cm}@{}}
            \toprule
            \textbf{Biblioteca} & \textbf{Estilo} & \textbf{Complejidad} & \textbf{Rendimiento} & \textbf{Integración Gráfica} & \textbf{Multiplataforma} & \textbf{Adecuación al Proyecto} \\
            \midrule
            \rowcolor{yellow!25}
            \textbf{PyQt} & Profesional & \color{red!40!orange}{\textasciitilde} & \color{green}{\checkmark} & \color{green}{\checkmark} & \color{green}{\checkmark} & \color{green}{\checkmark} \\
            \midrule
            \textbf{Tkinter} & Anticuado & \color{green}{\checkmark} & \color{red!40!orange}{\textasciitilde} & \color{green}{\checkmark} & \color{green}{\checkmark} & \color{yellow}{\textasciitilde} \\
            \midrule
            \textbf{Streamlit} & Moderno & \color{green}{\checkmark} & \color{red!40!orange}{\textasciitilde} & \color{green}{\checkmark} & \color{red!40!orange}{\textasciitilde} & \color{red}{\xmark} \\
            \midrule
            \textbf{PySide} & Profesional & \color{red!40!orange}{\textasciitilde} & \color{green}{\checkmark} & \color{green}{\checkmark} & \color{green}{\checkmark} & \color{green}{\checkmark} \\
            \midrule
            \textbf{Dear PyGui} & Herramientas & \color{red!40!orange}{\textasciitilde} & \color{green}{\checkmark} & \color{red!40!orange}{\textasciitilde} & \color{green}{\checkmark} & \color{red}{\xmark} \\
            \bottomrule
        \end{tabular}
    \end{adjustbox}
    \smallskip
    \vspace{0cm}\\
    \tiny{*Símbolos: \checkmark~(Bueno), \textasciitilde~(Regular), \xmark~(No adecuado).}\\
    \tiny{**Adecuación al Proyecto: \checkmark~(Excelente), \textasciitilde~(Adecuada), \xmark~(No ideal).}
\end{frame}

\begin{frame}{Anexo \thesection~: Enfoques principales en problemas de N-Cuerpos}
    \centering
    \begin{table}[H]
    \centering
    \caption[Enfoques en $n$ cuerpos]{\small Comparativa de enfoques principales a la hora de resolver problemas de $n$ cuerpos}%
    \label{tab:EnfoquesNCuerpos}
    \vspace{-0.2cm}
    \begin{adjustbox}{max width=1.2\textwidth,max height=0.35\textheight, keepaspectratio}
        \begin{tabular}{@{}p{0.3\textwidth}p{0.5\textwidth}cc@{}}
            \toprule
            \textbf{Enfoque} & \textbf{Descripción} & \textbf{Complejidad} & \textbf{Referencia} \\
            \midrule
            Solución Analítica (n=2) &
            Método que resuelve el problema de Kepler para dos cuerpos.

            Proporciona soluciones exactas para sistemas de dos objetos. &
            $O(1)$ &~\cite{newton1687} \\
            \midrule
            Integración Numérica &
            Utiliza métodos como Runge-Kutta para simular movimientos.

            Permite aproximaciones numéricas para sistemas complejos. &
            $O(n^2)$ &~\cite{Orlov2017} \\
            \midrule
            Algoritmo de Barnes-Hut &
            Método de agrupamiento jerárquico para reducir complejidad computacional.

            Mejora la eficiencia en simulaciones de múltiples cuerpos. &
            $O(n \log n)$ &~\cite{Barnes1986} \\
            \bottomrule
        \end{tabular}
    \end{adjustbox}
\end{table}
\end{frame}

\begin{frame}{Anexo \thesection~: Selección de Integradores Simplécticos}
    \centering
    \captionof{table}{Evaluación de Integradores Simplécticos Relevantes}%
    \label{tab:integrators_beamer_short}
    \vspace{-0.1cm}
    \begin{adjustbox}{max width=\textwidth, max height=0.32\textheight, keepaspectratio}
        \renewcommand{\arraystretch}{1.4}
        \begin{tabular}{
            >{\bfseries\raggedright}p{3.5cm} % Variante
            >{\centering\arraybackslash}p{4.5cm} % Eficiencia Kepleriana
            >{\centering\arraybackslash}p{4cm} % Manejo Encuentros
            >{\centering\arraybackslash}p{4cm} % Complejidad Implementación
        }
            \toprule
            \textbf{Integrador} & \textbf{Eficiencia en Sistemas Keplerianos Dominados} & \textbf{Manejo de Encuentros Cercanos} & \textbf{Complejidad del Método Base} \\
            \midrule
            Leapfrog/Verlet & Moderada & \xmark~(Requiere regularización)  & Baja \\
            \midrule
            \rowcolor{yellow!30}
            Wisdom-Holman (WHFast)  & \textbf{\cmark~\cmark~(Muy Alta)} & \textbf{\qmark~(Requiere hibridación)} & \textbf{Moderada} \\
            \midrule
            Orden Superior (Yoshida)& Variable & \xmark~(Puede ser inestable) & Alta \\
            \midrule
            \rowcolor{yellow!30}
            Híbridos (con WHFast) & Alta (base WHFast) & \cmark~(Propósito principal) & Alta (Combinación) \\
            \bottomrule
        \end{tabular}
    \end{adjustbox}
    \smallskip
    \vspace{0.2cm}
    \footnotesize\\
    \textbf{Leyenda:} \cmark~\cmark~Muy Bueno/Alto, \cmark~Bueno/Moderado, \qmark~Depende/Requiere Adicional, \xmark~Bajo/No Directo
\end{frame}

% \begin{frame}{Estado del arte}
%     \centering
%     \captionof{table}{Comparativa contra soluciones disponibles}%
%     \label{tab:arte}
%     \vspace{-0.1cm}
%     \begin{adjustbox}{max width=0.9\textwidth,max height=0.7\textheight, keepaspectratio}
%         \renewcommand{\arraystretch}{1.3}
%             \begin{tabular}{@{}>{\bfseries}p{0.35\textwidth} p{0.4\textwidth} p{0.20\textwidth} p{0.15\textwidth} p{0.35\textwidth}@{}}
%             \toprule
%             \textbf{Producto o metodo} & \textbf{Características} & \textbf{Escalabilidad} & \textbf{Usa IA} & \textbf{Cambios dinamicos} \\
%             \midrule
%             ode\_num\_int & Framework C++11 modular para EDOs; orientado a pruebas de integradores. & Media & No & No \\
%             Representación Geométrica & Quadtrees/Octrees para geometría eficiente, no simula dinámica. & Alta & No & No \\
%             Método n-NNN & Simulación con n-vecinos y cirugía Hamiltoniana; inspirado en IA.\ & Alta & Sí & No \\
%             PKDGRAV3 & Hidrodinámica sin malla (MFM/MFV); vecinos adaptativos. & Alta & No & No \\
%             \bottomrule
%             \end{tabular}
%     \end{adjustbox}
%     \smallskip
% \end{frame}
%
%
%
% \begin{frame}{Estado del arte}
%     \centering
%     \captionof{table}{Comparativa contra soluciones disponibles}%
%     \label{tab:arte}
%     \vspace{-0.1cm}
%     \begin{adjustbox}{max width=0.9\textwidth,max height=0.7\textheight, keepaspectratio}
%         \renewcommand{\arraystretch}{1.3}
%             \begin{tabular}{@{}>{\bfseries}p{0.35\textwidth} p{0.4\textwidth} p{0.20\textwidth} p{0.15\textwidth} p{0.35\textwidth}@{}}
%             \toprule
%             \textbf{Producto o metodo} & \textbf{Características} & \textbf{Escalabilidad} & \textbf{Usa IA} & \textbf{Cambios dinamicos} \\
%             \midrule
%             SPH/N-body Híbrido & Interacciones gas-estrella con árbol Barnes-Hut y pasos bloque. & Alta & No & No \\
%             Integrador Simpléctico & Orden 2+, reversible; ideal para colisiones. & Media & No & No \\
%             Solver TPM & Combina PM y Tree según densidad local; altamente paralelizado. & Alta & No & No \\
%             Algoritmo TPM & Descomposición por densidad; integración multi-escala. & Alta & No & No \\
%             \bottomrule
%             \end{tabular}
%     \end{adjustbox}
%     \smallskip
% \end{frame}
%
%
% \begin{frame}{Estado del arte}
%     \centering
%     \captionof{table}{Comparativa contra soluciones disponibles}%
%     \label{tab:arte}
%     \vspace{-0.1cm}
%     \begin{adjustbox}{max width=0.9\textwidth,max height=0.7\textheight, keepaspectratio}
%         \renewcommand{\arraystretch}{1.3}
%             \begin{tabular}{@{}>{\bfseries}p{0.35\textwidth} p{0.4\textwidth} p{0.20\textwidth} p{0.15\textwidth} p{0.35\textwidth}@{}}
%             \toprule
%             \textbf{Producto o metodo} & \textbf{Características} & \textbf{Escalabilidad} & \textbf{Usa IA} & \textbf{Cambios dinamicos} \\
%             \midrule
%             REBOUND & Modular; incluye varios integradores, colisiones y condiciones frontera. & Alta & No & No \\
%             Estabilidad Planetaria & Estudio con REBOUND;\ Gini vs.\ inestabilidad. No eventos internos. & Alta & No & No \\
%             \midrule
%             \rowcolor{yellow!20}
%             \textbf{Solución Propuesta} & \textbf{Combina FMM/Barnes-Hut para cálculo gravitacional eficiente con Algoritmos Bioinspirados para la \textit{optimización y ajuste dinámico} de parámetros (masa) durante la simulación.}  & \textbf{Alta} & \textbf{SÍ} & \textbf{SÍ} \\
%             \bottomrule
%             \end{tabular}
%     \end{adjustbox}
%     \smallskip
% \end{frame}