\section[Comparativas]{Tablas Comparativas del Marco Teórico}

\begin{frame}{Anexo \thesection~: Lenguaje de Programación}
    \centering
    \captionof{table}{Comparativa de Lenguaje de Programación}%
    \label{tab:comparativa}
    \vspace{-0.1cm}
    \begin{adjustbox}{max width=0.9\textwidth,max height=0.7\textheight, keepaspectratio}
        \renewcommand{\arraystretch}{1.3}
            \begin{tabular}{@{}>{\bfseries}p{0.5\textwidth} p{0.6\textwidth} >{\columncolor{yellow!30}}p{0.6\textwidth}@{}}
            \toprule
            \textbf{Característica} & \textbf{C++} & \textbf{Python} \\
            \midrule
            Velocidad de Ejecución & Máxima. Ideal para cálculos intensivos. & Menor. \\
            Velocidad de Desarrollo & Lento y verboso. & Rápido, claro y flexible para iterar. \\
            Bibliotecas Científicas & Potentes, pero más complejas de integrar. & Vastas y accesibles. \\
            Visualización & Requiere herramientas externas. & Fácil con Matplotlib, Plotly o Bokeh. \\
            Integración con REBOUND & Directa con linking C. & Interfaz oficial en Python lista para usar. \\
            Algoritmos Bioinspirados & Óptimo si se implementan desde cero. & Fáciles de coordinar o conectar con código externo. \\
            Gestión de Memoria & Manual. Mayor control. & Automática. Simplifica el desarrollo. \\
            Curva de Aprendizaje & Alta. & Baja a moderada.. \\
            Prototipado / Experimentación & Lento. & Ágil. \\
            Depuración & Complejo. & Sencillo. \\
            Enfoque del Proyecto & Control total, pero puede complicar. & Permite centrarse en la solución y resultados. \\
            \bottomrule
            \end{tabular}
    \end{adjustbox}
    \smallskip
\end{frame}


\begin{frame}{Anexo \thesection~: Comparativa de Bibliotecas N-Cuerpos}
    \centering
    \captionof{table}{Comparativa extensa de bibliotecas para simulación N-cuerpos.}%
    \label{tab:nbody-comparativa-beamer}
    \vspace{-0.1cm} % Espacio como en el ejemplo
    \begin{adjustbox}{max width=\textwidth, max height=0.7\textheight, keepaspectratio} % Aumentamos max height
        % Ajustes de la tabla original
        \renewcommand{\arraystretch}{1.2} % Reducido un poco por si acaso, puedes probar 1.3
        \scriptsize % o \tiny si es necesario más pequeño
        %\rowcolors{2}{gray!10}{white} % Colores de fila alternos, empezando por la segunda fila de datos
        \begin{tabular}{
            @{}>{\bfseries}p{0.4\textwidth}
            >{\columncolor{yellow!30}}p{0.3\textwidth}
            p{0.3\textwidth}
            p{0.3\textwidth}
            p{0.3\textwidth}
            p{0.3\textwidth}@{}}
        \toprule
        \textbf{Característica} & \textbf{REBOUND} & \textbf{PKDGRAV3} & \textbf{AMUSE} & \textbf{NBody (Python)} & \textbf{PyGaia} \\
        \midrule

        Lenguaje Principal / Interfaz & C (Python API) & C++ & Python & Python & Python \\

        Enfoque Principal & Colisional/no colisional, sistemas N-cuerpos & Cosmología a gran escala & Framework multipropósito & Simulaciones educativas & Análisis Gaia \\

        Tipos de Problemas & Planetas, cúmulos, anillos & Galaxias, materia oscura & Multifísica astrofísica & Sistemas pequeños & Dinámica galáctica \\

        Algoritmos de Gravedad & Barnes-Hut/Suma directa & TreePM & Hermite/Tree/SPH & Suma directa & Potenciales analíticos \\

        Manejo de Colisiones & Sí (esferas duras) & No & Depende del backend & No & N/A \\

        Integradores Numéricos & WHFast/ IAS15/ Leapfrog & Leapfrog KDK & Hermite/ Symplectic & RK/ Leapfrog & SciPy ODE \\

        Hidrodinámica (SPH/Gas) & No  & Sí & Sí & No & No \\

        Paralelización & MPI/OpenMP & MPI & MPI frameworks & Multiprocessing & CPU básica \\

        Flexibilidad/Modularidad & Módulos intercambiables & Enfoque cosmología & Interoperabilidad & Implementación-dependiente & Centrado en Gaia \\

        Facilidad de Uso (Python) & Excelente docs/API & N/A (C++) & Compleja (múltiples backends) & Variable & Astronomer-friendly \\

        Comunidad/Mantenimiento & Activo desarrollo & Cosmología activa & Colaborativo & Individual & Soporte Gaia \\

        Idoneidad para el Proyecto & \textbf{Óptimo:} • Soporta 2 cuerpos • Python • Modular & \textbf{Inadecuado:} Escala/física diferente & \textbf{Complejidad excesiva} para necesidades simples & \textbf{Muy básico} para requisitos & \textbf{Enfoque observacional} no aplicable \\
        \bottomrule
        \end{tabular}
    \end{adjustbox}
    \smallskip % Espacio como en el ejemplo
\end{frame}

\begin{frame}{Anexo \thesection~: Algoritmos de Optimización}
    \centering
    \captionof{table}{Comparativa de Bibliotecas de Optimización}%
    \label{tab:comparativa}
    \vspace{-0.1cm}
    \begin{adjustbox}{max width=0.9\textwidth,max height=0.4\textheight, keepaspectratio}
        \renewcommand{\arraystretch}{1.3}
            \begin{tabular}{@{}>{\bfseries}p{0.2\textwidth} p{0.5\textwidth} p{0.5\textwidth} p{0.4\textwidth}@{}}
            \toprule
            \textbf{Biblioteca} & \textbf{Enfoque Clave / Fortaleza Principal} & \textbf{Ideal Para (Contexto del Proyecto)} & \textbf{Complejidad / Flexibilidad} \\
            \midrule
            \rowcolor{yellow!25}
            \texttt{pymoo} & Framework moderno y completo para optimización \textbf{multiobjetivo} (y mono). Amplia gama de algoritmos. & Problemas complejos, si se requieren múltiples objetivos o algoritmos robustos mono-objetivo. & Moderada / Alta \\
            \texttt{DEAP} & Máxima \textbf{flexibilidad} para construir algoritmos evolutivos desde cero. & Experimentación profunda con la estructura interna de los algoritmos (ej. GA personalizado). & Alta / Muy Alta \\
            \texttt{Platypus} & \textbf{Optimización multiobjetivo fácil de usar} con algoritmos estándar (NSGA-II, SPEA2). & Implementación rápida de MOEAs conocidos, buen punto de partida para multiobjetivo. & Baja-Moderada / Moderada \\
            \texttt{Inspyred} & Framework versátil para varios algoritmos evolutivos y metaheurísticas. & Exploración de diferentes tipos de algoritmos evolutivos si las opciones más nuevas no son prioritarias. & Moderada / Alta \\
            \texttt{Nevergrad} & \textbf{Optimización sin derivadas (caja negra)}; ideal para funciones costosas/ruidosas. & Cuando la función objetivo (simulación + LE) es compleja y sin gradientes fáciles. & Moderada / Alta \\
            \texttt{PyGMO/PaGMO} & \textbf{Alto rendimiento y paralelización} (backend C++) para problemas globales complejos. & Simulaciones muy costosas donde la paralelización es crítica para la eficiencia. & Moderada-Alta / Alta \\
            \bottomrule
            \end{tabular}
    \end{adjustbox}
    \smallskip
\end{frame}

\begin{frame}{Anexo \thesection~: Bibliotecas de Visualización}
    \centering
    \captionof{table}{Comparativa de Bibliotecas de Visualización}%
    \label{tab:comparativa}
    \vspace{-0.1cm}
    \begin{adjustbox}{max width=\textwidth,max height=0.35\textheight, keepaspectratio}
        \begin{tabular}{@{}>{\bfseries}p{0.35\textwidth} >{\columncolor{yellow!30}}p{0.3\textwidth} p{0.3\textwidth} p{0.3\textwidth} p{0.3\textwidth}@{}}
            \toprule
            \textbf{Característica} & \textbf{Matplotlib} & \textbf{Plotly} & \textbf{Bokeh} & \textbf{Seaborn} \\
            \midrule
            Uso Principal/Fortaleza & Gráficos científicos. & Interactividad web. & Interactividad web. & Estadísticas. \\
            Animación & Muy capaz. Ideal para simulaciones. & Buena para animaciones web. & Buena para datos en streaming. & Limitada. \\
            Gráficos Estáticos & Excelente. & Buena. & Buena. & Regular. \\
            Interactividad & Básica. & Alta. & Alta. & Básica. \\
            Interfaz & Simple. & Web & Web. & N/A. \\
            Calidad Publicación & Muy alta. & Alta. & Alta. & Alta. \\
            Trayectorias 2D/3D & Directa. & Clara y efectiva. & Buena. & N/A. \\
            Curva Aprendizaje (anim.) & Moderada. & Moderada. & Moderada. & Moderada. \\
            Estética por Defecto & Funcional. & Moderna. & Flexible. & Mejorada estadísticas. \\
            Comunidad/Docs & Muy extensa. & En crecimiento. & Buena. & Buena. \\
            Adecuación Proyecto & Excelente: animación, análisis. & Interactividad web. & Prioridad web. & Análisis complementario. \\
            \bottomrule
        \end{tabular}
    \end{adjustbox}
    \smallskip
\end{frame}

\begin{frame}{Anexo \thesection~: Comparativa de Bibliotecas GUI}
    \centering
    \captionof{table}{Comparativa de bibliotecas Python para Interfaces Gráficas.}%
    \label{tab:gui-comparativa-beamer}
    \vspace{-0.1cm}
    \begin{adjustbox}{max width=\textwidth, max height=0.8\textheight, keepaspectratio}
        \renewcommand{\arraystretch}{1.2}
        \scriptsize
        %\rowcolors{2}{gray!10}{white}
        \begin{tabular}{
          @{}>{\bfseries}p{0.35\textwidth}
          >{\columncolor{yellow!30}}p{0.25\textwidth}
          p{0.25\textwidth}
          p{0.25\textwidth}
          p{0.25\textwidth}
          p{0.25\textwidth}@{}}
        \toprule
        \textbf{Característica} & \textbf{PyQt} & \textbf{Tkinter} & \textbf{Streamlit} & \textbf{PySide} & \textbf{Dear PyGui} \\
        \midrule
        Toolkit subyacente & Qt (C++) & Tcl/Tk & React (Web) & Qt (C++) & GPU (ImGui-like) \\
        Estilo & Profesional, personalizable & Anticuado, editable & Moderno, limpio & Igual a PyQt & Herramientas/juegos \\
        Complejidad de desarrollo & Media-Alta & Baja-Media & Muy baja & Igual a PyQt & Media \\
        Curva de aprendizaje & Moderada & Baja & Muy baja & Igual a PyQt & Requiere nuevo concepto \\
        Rendimiento & Muy bueno, optimizado & Bueno en GUIs simples & Bueno, depende del navegador & Muy bueno, igual a PyQt & Excelente, acelerado GPU\\
        Widgets disponibles & Extensa colección madura & Básica, útil & Limitada, centrada en datos & Igual a PyQt & Buena para herramientas\\
        Layouts & Extensos, Qt Designer & Pack/grid/place & Automáticos, poco control        & Igual a PyQt & Control total programático \\
        Integración gráfica & Excelente con Matplotlib, etc. & Buena con Matplotlib & Excelente para Plotly y Altair & Igual a PyQt & Posible, requiere ajustes      \\
        Multihilo & Señales/slots robustos & Limitado, no thread-safe nativo  & Abstracción interna & Igual a PyQt & Usuario maneja concurrencia \\
        Multiplataforma & Win/macOS/Linux  & Win/macOS/Linux & Navegador (web) & Win/macOS/Linux & Win/macOS/Linux \\
        Licencia & GPL/comercial (LGPL en PyQt5) & Lib.\ estándar (libre) & Apache 2.0 (libre) & LGPL (comercial viable) & MIT (libre) \\
        Comunidad y documentación & Amplia y activa & Amplia  & Activa & Activa y sólida & Creciente y entusiasta         \\
        Idoneidad para el proyecto& Excelente: robusto, flexible, GUI + Matplotlib embebido & Adecuada para GUI simple & Menos ideal: mejor para dashboards web & Excelente alternativa LGPL & Paradigma distinto, menos directo \\
        \bottomrule
        \end{tabular}
    \end{adjustbox}
    \smallskip
\end{frame}

\begin{frame}{Anexo \thesection~: Enfoques principales en problemas de N-Cuerpos}
    \centering
    \begin{table}[H]
    \centering
    \caption[Enfoques en $n$ cuerpos]{\small Comparativa de enfoques principales a la hora de resolver problemas de $n$ cuerpos}%
    \label{tab:EnfoquesNCuerpos}
    \vspace{-0.2cm}
    \begin{adjustbox}{max width=1.2\textwidth,max height=0.35\textheight, keepaspectratio}
        \begin{tabular}{@{}p{0.3\textwidth}p{0.5\textwidth}cc@{}}
            \toprule
            \textbf{Enfoque} & \textbf{Descripción} & \textbf{Complejidad} & \textbf{Referencia} \\
            \midrule
            Solución Analítica (n=2) &
            Método que resuelve el problema de Kepler para dos cuerpos.

            Proporciona soluciones exactas para sistemas de dos objetos. &
            $O(1)$ &~\cite{newton1687} \\
            \midrule
            Integración Numérica &
            Utiliza métodos como Runge-Kutta para simular movimientos.

            Permite aproximaciones numéricas para sistemas complejos. &
            $O(n^2)$ &~\cite{Orlov2017} \\
            \midrule
            Algoritmo de Barnes-Hut &
            Método de agrupamiento jerárquico para reducir complejidad computacional.

            Mejora la eficiencia en simulaciones de múltiples cuerpos. &
            $O(n \log n)$ &~\cite{Barnes1986} \\
            \bottomrule
        \end{tabular}
    \end{adjustbox}
\end{table}
\end{frame}

\begin{frame}{Anexo \thesection~: Selección de Integradores Simplécticos}
    \centering
    \captionof{table}{Evaluación de Integradores Simplécticos Relevantes}%
    \label{tab:integrators_beamer_short}
    \vspace{-0.1cm}
    \begin{adjustbox}{max width=\textwidth, max height=0.32\textheight, keepaspectratio}
        \renewcommand{\arraystretch}{1.4}
        \begin{tabular}{
            >{\bfseries\raggedright}p{3.5cm} % Variante
            >{\centering\arraybackslash}p{4.5cm} % Eficiencia Kepleriana
            >{\centering\arraybackslash}p{4cm} % Manejo Encuentros
            >{\centering\arraybackslash}p{4cm} % Complejidad Implementación
        }
            \toprule
            \textbf{Integrador} & \textbf{Eficiencia en Sistemas Keplerianos Dominados} & \textbf{Manejo de Encuentros Cercanos} & \textbf{Complejidad del Método Base} \\
            \midrule
            Leapfrog/Verlet & Moderada & \xmark~(Requiere regularización)  & Baja \\
            \midrule
            \rowcolor{yellow!30}
            Wisdom-Holman (WHFast)  & \textbf{\cmark~\cmark~(Muy Alta)} & \textbf{\qmark~(Requiere hibridación)} & \textbf{Moderada} \\
            \midrule
            Orden Superior (Yoshida)& Variable & \xmark~(Puede ser inestable) & Alta \\
            \midrule
            \rowcolor{yellow!30}
            Híbridos (con WHFast) & Alta (base WHFast) & \cmark~(Propósito principal) & Alta (Combinación) \\
            \bottomrule
        \end{tabular}
    \end{adjustbox}
    \smallskip
    \vspace{0.2cm}
    \footnotesize\\
    \textbf{Leyenda:} \cmark~\cmark~Muy Bueno/Alto, \cmark~Bueno/Moderado, \qmark~Depende/Requiere Adicional, \xmark~Bajo/No Directo
\end{frame}

% \begin{frame}{Estado del arte}
%     \centering
%     \captionof{table}{Comparativa contra soluciones disponibles}%
%     \label{tab:arte}
%     \vspace{-0.1cm}
%     \begin{adjustbox}{max width=0.9\textwidth,max height=0.7\textheight, keepaspectratio}
%         \renewcommand{\arraystretch}{1.3}
%             \begin{tabular}{@{}>{\bfseries}p{0.35\textwidth} p{0.4\textwidth} p{0.20\textwidth} p{0.15\textwidth} p{0.35\textwidth}@{}}
%             \toprule
%             \textbf{Producto o metodo} & \textbf{Características} & \textbf{Escalabilidad} & \textbf{Usa IA} & \textbf{Cambios dinamicos} \\
%             \midrule
%             ode\_num\_int & Framework C++11 modular para EDOs; orientado a pruebas de integradores. & Media & No & No \\
%             Representación Geométrica & Quadtrees/Octrees para geometría eficiente, no simula dinámica. & Alta & No & No \\
%             Método n-NNN & Simulación con n-vecinos y cirugía Hamiltoniana; inspirado en IA.\ & Alta & Sí & No \\
%             PKDGRAV3 & Hidrodinámica sin malla (MFM/MFV); vecinos adaptativos. & Alta & No & No \\
%             \bottomrule
%             \end{tabular}
%     \end{adjustbox}
%     \smallskip
% \end{frame}
%
%
%
% \begin{frame}{Estado del arte}
%     \centering
%     \captionof{table}{Comparativa contra soluciones disponibles}%
%     \label{tab:arte}
%     \vspace{-0.1cm}
%     \begin{adjustbox}{max width=0.9\textwidth,max height=0.7\textheight, keepaspectratio}
%         \renewcommand{\arraystretch}{1.3}
%             \begin{tabular}{@{}>{\bfseries}p{0.35\textwidth} p{0.4\textwidth} p{0.20\textwidth} p{0.15\textwidth} p{0.35\textwidth}@{}}
%             \toprule
%             \textbf{Producto o metodo} & \textbf{Características} & \textbf{Escalabilidad} & \textbf{Usa IA} & \textbf{Cambios dinamicos} \\
%             \midrule
%             SPH/N-body Híbrido & Interacciones gas-estrella con árbol Barnes-Hut y pasos bloque. & Alta & No & No \\
%             Integrador Simpléctico & Orden 2+, reversible; ideal para colisiones. & Media & No & No \\
%             Solver TPM & Combina PM y Tree según densidad local; altamente paralelizado. & Alta & No & No \\
%             Algoritmo TPM & Descomposición por densidad; integración multi-escala. & Alta & No & No \\
%             \bottomrule
%             \end{tabular}
%     \end{adjustbox}
%     \smallskip
% \end{frame}
%
%
% \begin{frame}{Estado del arte}
%     \centering
%     \captionof{table}{Comparativa contra soluciones disponibles}%
%     \label{tab:arte}
%     \vspace{-0.1cm}
%     \begin{adjustbox}{max width=0.9\textwidth,max height=0.7\textheight, keepaspectratio}
%         \renewcommand{\arraystretch}{1.3}
%             \begin{tabular}{@{}>{\bfseries}p{0.35\textwidth} p{0.4\textwidth} p{0.20\textwidth} p{0.15\textwidth} p{0.35\textwidth}@{}}
%             \toprule
%             \textbf{Producto o metodo} & \textbf{Características} & \textbf{Escalabilidad} & \textbf{Usa IA} & \textbf{Cambios dinamicos} \\
%             \midrule
%             REBOUND & Modular; incluye varios integradores, colisiones y condiciones frontera. & Alta & No & No \\
%             Estabilidad Planetaria & Estudio con REBOUND;\ Gini vs.\ inestabilidad. No eventos internos. & Alta & No & No \\
%             \midrule
%             \rowcolor{yellow!20}
%             \textbf{Solución Propuesta} & \textbf{Combina FMM/Barnes-Hut para cálculo gravitacional eficiente con Algoritmos Bioinspirados para la \textit{optimización y ajuste dinámico} de parámetros (masa) durante la simulación.}  & \textbf{Alta} & \textbf{SÍ} & \textbf{SÍ} \\
%             \bottomrule
%             \end{tabular}
%     \end{adjustbox}
%     \smallskip
% \end{frame}